\subsection{Na czym polega katastrofa ultrafioletowa?}

odpowiedź

\subsection{Jakie są eksperymentalne dowody tego, że światło istnieje, się emituje oraz jest absorbowane porcjami (kwantami)?}

odpowiedź

\subsection{Przedyskutuj zjawisko interferencji fal}

odpowiedź

\subsection{Czym są pakiety falowe? Problem normalizacji, przekształcenia pomiędzy przestrzenią położenia oraz przestrzenią pędu.}

odpowiedź

\subsection{Stany kwantowe, operatory oraz równanie Schrödingera w interpretacji Feynmana}

odpowiedź

\subsection{Twierdzenie Ehrenfesta}

odpowiedź

\subsection{Obserwable, równanie Schrödingera, zależne oraz niezależne od czasu, własne stany, własne energie}

odpowiedź

\subsection{Proste zagadnienia: swobodna cząstka}

odpowiedź

\subsection{Proste zagadnienia: potencjał w kształcie schodków, bariery, studni kwadratowej}
Szukamy rozwiązania równania Schrödingera dla ustalonych postaci potencjału niezależnego od czasu $V(x)$. Rozwiązujemy w tym celu równanie
\begin{equation*}
	-\frac{\hbar^2}{2m} \frac{\partial^2}{\partial x^2} \psi(x) + V(x)\psi(x) = E\psi(x).
\end{equation*}

\subsubsection*{Potencjał schodkowy}
\begin{equation*}
	V(x) = \begin{cases}
		0, &x < 0,\\
		V_0, &x \geq 0.
	\end{cases}
\end{equation*}
Rozwiążemy równanie osobno dla $x < 0$ i $x > 0$, a na końcu połączymy te rozwiązania tak, by mogło służyć do opisu ruchu cząstek między obszarami.

Dla $x < 0$ mamy równanie $\frac{\partial^2}{\partial x^2} \psi(x) + \frac{2mE}{\hbar^2}\psi(x) = 0,$ którego rozwiązaniem ogólnym jest 
\begin{equation*}
	\psi_1(x) = A \mathrm{e}^{i k_1 x} + B \mathrm{e}^{-i k_1 x}, \text{ gdzie } k_1 = \frac{\sqrt{2mE}}{\hbar}.
\end{equation*}

Dla $x > 0$ mamy równanie $\frac{\partial^2}{\partial x^2} \psi(x) + \frac{2m(E - V_0)}{\hbar^2}\psi(x) = 0,$ którego rozwiązaniem ogólnym jest 
\begin{equation*}
	\psi_2(x) = C \mathrm{e}^{i k_2 x} + D \mathrm{e}^{-i k_2 x}, \text{ gdzie } k_2= \frac{\sqrt{2m(E - V_0)}}{\hbar}.
\end{equation*}

Rozwiązanie zależeć będzie zatem od tego, po której stronie wysokości schodka jest $E$.
\begin{enumerate}
	\item Jeśli $0 < E < V_0$, to mamy \begin{equation*}
		\psi_2(x) = C \mathrm{e}^{-\kappa x} + D \mathrm{e}^{\kappa x}, \text{ gdzie } \kappa = \frac{\sqrt{2m(V_0 - E)}}{\hbar}.
	\end{equation*}
	Ponieważ drugi składnik rośnie nieograniczenie wraz ze wzrostem $x$, musi zachodzić $D = 0$. Stałe $A, B$ i $C$ możemy wyznaczyć z warunków ciągłości:
	\begin{equation*}
		\psi_1(0) = \psi_2(0), \quad \psi_1'(0) = \psi_2'(0).
	\end{equation*}
	Mamy więc 
	\begin{equation*}
		A + B = C, \quad i k_1 (A - B) = -\kappa C
	\end{equation*}
	Otrzymaliśmy układ dwóch równań liniowych na trzy nieznane zespolone amplitudy. Wyznaczymy $A$ i $B$ w zależności od $C$.
	\begin{equation*}
		A = \frac{k_1 + i\kappa}{2k_1}C, \quad B = \frac{k_1 - i\kappa}{2k_1}C.
	\end{equation*}
	Wynika stąd, że
	\begin{equation*}
		\frac{B}{A} = \mathrm{e}^{i\alpha}, \quad \alpha = 2\arctan\left(-\sqrt{\frac{V_0}{E} - 1}\right).
	\end{equation*}
	Fala odbita różni się od fali padającej przesunięciem fazowym $\alpha$, co wpływa na interferencję i kształt fali stojącej w obszarze $x<0$.
	Współczynnik odbicia to
	\begin{equation*}
		R = \frac{|B|^2}{|A|^2} = |\mathrm{e}^{i\alpha}|^2 = 1,
	\end{equation*}
	a zatem zachodzi całkowite odbicie od schodka i nie ma przenikania do obszaru o wyższym potencjale.
	
	\item Jeśli $E > V_0$, to $D = 0$. Jest tak, ponieważ $D$ odpowiada amplitudzie fali biegnącej od prawej do lewej, natomiast dla $x>0$ nie ma możliwości odbicia, ponieważ potencjał jest stały (nie ma kolejnej bariery). Pozostałe stałe wyznaczamy z warunków ciągłości:
	\begin{equation*}
		A + B = C, \quad ik_1 (A - B) = ik_2 C,
	\end{equation*}
	a zatem mamy
	\begin{equation*}
		B = \frac{k_1 - k_2}{k_1 + k_2}A, \quad C = \frac{2k_1 }{k_1 + k_2}A.
	\end{equation*}
	Współczynniki odbicia i przejścia to
	\begin{align*}
		R &= \frac{|B|^2}{|A|^2} = \left(\frac{k_1 - k_2}{k_1 + k_2}\right)^2 = \left(\frac{1 - \sqrt{1 - \frac{V_0}{E}}}{1 + \sqrt{1 - \frac{V_0}{E}}}\right)^2,\\
		T &= \frac{k_2}{k_1}\frac{|C|^2}{|A|^2} = \frac{4k_1k_2}{(k_1 + k_2)^2} = \frac{4\sqrt{1 + \frac{V_0}{E}}}{\left(1 - \sqrt{1 - \frac{V_0}{E}}\right)^2}.
	\end{align*}
	
	\item Jeśli $E < 0$, równanie nie ma rozwiązań. Być może wymaga to dodatkowego wyjaśnienia, ale nie wiem jak ono dokładnie brzmi.
\end{enumerate}

\subsubsection*{Bariera potencjału}
\begin{equation*}
	V(x) = \begin{cases}
		0, &x \leq 0,\\
		V_0, &0<x<a,\\
		0, &x\geq a
	\end{cases}
\end{equation*}
Przypadek energii ujemnej wykluczamy tak jak w poprzednim przykładzie. Musimy więc znów rozważyć dwa przypadki: $0<E<V_0$ oraz $E>V_0$.

Zaczniemy od pierwszego z nich. Ponownie znajdujemy rozwiązanie ogólne równania Schrödingera na wszystkich trzech przedziałach.
\begin{equation*}
	\psi(x) = \begin{cases}
		A\mathrm{e}^{ikx} + B\mathrm{e}^{-ikx}, &x<0, \quad k = \frac{\sqrt{2mE}}{\hbar},\\
		C\mathrm{e}^{ikx}, & x>a,\\
		F\mathrm{e}^{\kappa x} + G\mathrm{e}^{-\kappa x}, &0<x<a, \quad \kappa = \frac{\sqrt{2m(V_0 - E)}}{\hbar}.
	\end{cases}
\end{equation*}
Zauważmy, że nie ma składnika $D\mathrm{e}^{-ikx}$, ponieważ w obszarze $x>a$ spotkamy tylko to, co przeszło przez barierę --- nie będzie tam więc cząstek poruszających się ,,w lewo''.

Stałe ponownie możemy wyznaczyć z warunków ciągłości $\psi(x)$ i $\psi'(x)$ w $0$ i $a$. Można pokazać, że
\begin{equation*}
	R = \frac{1}{1 + \frac{4E(V_0 - E)}{V_0^2 \sinh^2(\kappa a)}}, \quad T = \frac{1}{1 + \frac{V_0^2 \sinh^2(\kappa a)}{4E(V_0 - E)}}.
\end{equation*}

Jeżeli zaś $E>V_0$, to
\begin{equation*}
	\psi(x) = \begin{cases}
		A\mathrm{e}^{ikx} + B\mathrm{e}^{-ikx}, &x<0, \quad k = \frac{\sqrt{2mE}}{\hbar},\\
		C\mathrm{e}^{ikx}, & x>a,\\
		F\mathrm{e}^{ik'x} + G\mathrm{e}^{-ik'x}, &0<x<a, \quad k' = \frac{\sqrt{2m(E-V_0)}}{\hbar}.
	\end{cases}
\end{equation*}
Prowadzi to do następujących wzorów na współczynniki odbicia i przejścia:
\begin{equation*}
		R = \frac{1}{1 + \frac{4E(E-V_0)}{V_0^2 \sin^2(k'a)}}, \quad T = \frac{1}{1 + \frac{V_0^2 \sin^2(k' a)}{4E(E-V_0)}}.
\end{equation*}

\subsubsection*{Studnia kwadratowa:}
Nieskończona studnia potencjału to model idealizowany, w którym cząstka jest całkowicie
uwięziona między nieprzekraczalnymi barierami potencjału.

\begin{equation*}
	V(x) = \begin{cases}
		0, &|x|<a,\\
		\infty, &|x|\geq a.
	\end{cases}
\end{equation*}
Funkcja falowa musi zerować się na granicach studni i poza nią, ponieważ cząstka nie może znajdować się poza nieskończonymi barierami:
\begin{equation*}
	\psi(x) = 0 \text{ dla } |x| \geq a.
\end{equation*}
Wewnątrz studni równanie Schrödingera opisuje swobodną cząstkę, ale z nałożonymi wa-
runkami brzegowymi wymuszającymi dyskretną strukturę stanów energetycznych
\begin{equation*}
	\begin{cases}
		-\frac{\hbar^2}{2m}\frac{\partial^2}{\partial x^2}\psi(x) = E\psi(x), &|x|<a\\
		\psi(a) = \psi(-a) = 0, & \text{warunki brzegowe.}
	\end{cases}
\end{equation*}
Rozwiązaniem ogólnym równania jest funkcja \begin{equation*}
	\psi(x) = A\cos(kx) + B\sin(kx), \qquad k = \frac{\sqrt{2mE}}{\hbar^2}.
\end{equation*}
Uwzględnienie warunków brzegowych prowadzi do układu
\begin{equation*}
	\begin{cases}
		A\cos(ka) &= 0,\\
		B\sin(ka) &= 0.
	\end{cases}
\end{equation*}
Ponieważ $\cos$ i $\sin$ nie mogą być jednocześnie zerami, otrzymujemy dwie klasy rozwiązań, co odpowiada symetrii funkcji falowej względem środka studni.

Przypadek I:
\begin{equation*}
	\begin{cases}
		B = 0\\
		\cos(ka) = 0
	\end{cases}
	 \implies
	\begin{cases}
		B = 0\\
		k = k_n = n \frac{\pi}{2a}, n = 1, 3, 5, \dots
	\end{cases}
\end{equation*}
Mamy więc $\psi_n(x) = A_n \cos(k_n x)$. Normalizacja funkcji falowej wymusza $A_n = \frac{1}{\sqrt{a}}$:
\begin{equation*}
	\int_{-a}^{a} |\psi(x)|^2 \mathrm{d}x = 1 \implies A_n^2 \int_{-a}^{a} \cos^2(k_nx) \mathrm{d}x = 1 \implies A_n^2 a = 1 \implies A_n = \frac{1}{\sqrt{a}}
\end{equation*}

Przypadek II:
\begin{equation*}
	\begin{cases}
		A = 0\\
		\sin(ka) = 0
	\end{cases}
	\implies
	\phi_n(x) = \frac{1}{\sqrt{a}} \sin(k_n x), \quad n = 2, 4, 6, \dots.
\end{equation*}

Spektrum energii jest dyskretne i rośnie z kwadratem liczby kwantowej n, co oznacza, że
cząstka może zajmować tylko określone poziomy energetyczne.
\begin{equation*}
	E_n = \frac{\hbar^2k_n^2}{2m} = \frac{\hbar^2\pi^2}{2mL^2}n^2,
\end{equation*}
gdzie $L = 2a$ jest szerokością studni. Wartości $k_n$ są ściśle powiązane
z poziomami energii i określają kształt funkcji falowej w studni.
\subsection{Proste zagadnienia: oscylator harmoniczny}

Szukamy rozwiązań równania
\begin{equation*}
	-\frac{\hbar^2}{2m} \frac{\partial^2}{\partial x^2} \psi(x) + \frac12kx^2\psi(x) = E\psi(x).
\end{equation*}
Wbrew nazwie podsekcji, rozwiązanie go nie jest prostym zagadnieniem. Okazuje się, że istnieje rodzina rozwiązań postaci
\begin{equation*}
	\psi_n(x) = N_n H_n(\alpha x) \mathrm{e}^{-\frac{(\alpha x)^2}{2}},
\end{equation*} 
gdzie $\alpha = \sqrt[4]{\frac{km}{\hbar}}$, $H_n$ to wielomiany Hermite'a, natomiast $N_n$ jest stałą normującą, równą $\sqrt{\frac{\alpha}{\sqrt{\pi} 2^n n!}}$.

Odpowiadają im wartości własne postaci $E_n = \left(n + \frac12\right) \hbar \sqrt{\frac{k}{m}}$. Dla innych wartości energii funkcja falowa zmierza do nieskończoności przy $|x| \to \infty$.
\subsection{Formalizm mechaniki kwantowej, postulaty}
\begin{enumerate}
	\item Postulat I: Wektor stanu
	
	W każdej chwili czasu $t$ stan układu fizycznego jest określony przez wektor $\ket{\psi(t)}$ należący do pewnej przestrzeni Hilberta $\mathcal{H}$.
	
	\item Postulat II: obserwable
	
	Każdej mierzalnej wielkości fizycznej $\mathcal{A}$ odpowiada obserwabla $\hat{A}$ działająca w $\mathcal{H}$ (operator hermitowski).
	
	\item Postulat III: wyniki pomiarów --- wartości własne obserwabli
	
	Jedynym dopuszczalnym wynikiem pomiaru wielkości fizycznej $\mathcal{A}$ może być któraś z wartości własnych obserwabli $\hat{A}$.
	
	\item Postulat IV: prawdopodobieństwo wyników pomiarowych
	
	Prawdopodobieństwo tego, że w wyniku pomiaru wielkości fizycznej $\mathcal{A}$ w układzie opisanym wektorem stanu $\ket{\psi}$ otrzymamy wartość własną $a_n$ wynosi $|\braket{\phi_n}{\psi}|^2$, gdzie $\ket{\phi_n}$ jest wektorem własnym odpowiadającym wartości własnej $a_n$.
	
	\item Postulat V: redukcja (kolaps) wektora stanu
	
	Jeśli w układzie fizycznym opisanym stanem $\ket{\psi}$ dokonamy pomiaru wielkości fizycznej $\mathcal{A}$ otrzymując $a_n$, jedną z wartości własnych obserwabli $\hat{A}$, to po pomiarze stanem układu jest unormowany rzut stanu $\ket{\psi}$ na (unormowany) wektor własny $\ket{\phi_n}$ odpowiadający zmierzonej wartości własnej.
	\begin{equation*}
		\ket{\psi} \xrightarrow{\text{pomiar } a_n} \ket{\phi_n}\frac{\braket{\phi_n}{\psi}}{\sqrt{|\braket{\phi_n}{\psi}|^2}}.
	\end{equation*}
	
	Innymi słowy mówimy, że w wyniku pomiaru następuje redukcja (kolaps) stanu $\ket{\psi}$ do stanu $\ket{\phi_n}$.
	
	\item Postulat VI: ewolucja w czasie --- równanie Schrödingera
	
	Stan $\ket{\psi(t)}$ układu fizycznego ewoluuje w czasie zgodnie z równaniem Schrödingera
	\begin{equation*}
		i\hbar\frac{d}{dt} \ket{\psi(t)} = \hat{H}(t)\ket{\psi(t)},
	\end{equation*}
	gdzie hamiltonian $\hat{H}(t)$ jest obserwablą odpowiadającą całkowitej energii układu. 
\end{enumerate}

\subsection{Klasy i własności operatorów. Komutatory}

W mechanice kwantowej operatory reprezentują obserwowalne wielkości fizyczne, takie jak pozycja czy pęd. Operatory działają na przestrzeni stanów (np. w przestrzeni Hilberta) i mogą mieć różne własności: być hermitowskie (samosprzężone), jednostkowe czy projekcyjne. Hermitowskie operatory odpowiadają mierzalnym wartościom rzeczywistym.

Klasa operatorów określa ich charakter (np. ograniczone, nieograniczone). Ważnym pojęciem są komutatory operatorów
\[
[A, B] = AB - BA.
\]
Jeśli
\[
[A, B] = 0,
\]
to operatory się komutują, co oznacza, że można jednocześnie mierzyć odpowiadające im wielkości z pełną precyzją. Niezerowy komutator wskazuje na fundamentalne ograniczenia pomiarowe, jak w przypadku zasady nieoznaczoności Heisenberga.

\subsection{Zasada nieoznaczoności Heisenberga}

Zasada nieoznaczoności Heisenberga wyraża fundamentalne ograniczenie precyzji, z jaką można jednocześnie znać wartości pewnych par wielkości fizycznych, np. położenia \( \hat{x} \) i pędu \( \hat{p} \). Formalnie wyraża się to nierównością

\[
\Delta x \, \Delta p \geq \frac{\hbar}{2},
\]

gdzie \( \Delta x \) i \( \Delta p \) to odchylenia standardowe pomiarów operatorów położenia i pędu, a \( \hbar \) to zredukowana stała Plancka.

Ta zasada wynika z faktu, że operatory położenia i pędu nie komutują, tzn.

\[
[\hat{x}, \hat{p}] = i \hbar,
\]

co implikuje, że nie istnieje wspólny zbiór własnych wektorów obu operatorów, a więc nie można jednocześnie przypisać im dokładnych wartości.

\subsection{Operator momentu pędu, uogólniony operator momentu pędu, operator spinu: własne funkcje i wartości}


Operator momentu pędu \(\hat{\mathbf{L}} = (\hat{L}_x, \hat{L}_y, \hat{L}_z)\) opisuje moment pędu orbitalnego cząstki. Składowe operatora spełniają następujące relacje komutacyjne:

\[
[\hat{L}_i, \hat{L}_j] = i \hbar \epsilon_{ijk} \hat{L}_k,
\]

gdzie \(\epsilon_{ijk}\) to symbol Levi-Civity, a \(i, j, k \in \{x,y,z\}\).

Operator kwadrat momentu pędu \(\hat{L}^2 = \hat{L}_x^2 + \hat{L}_y^2 + \hat{L}_z^2\) oraz składowa \(\hat{L}_z\) mają wspólny układ własnych funkcji \(|l, m\rangle\), dla których zachodzą własności:

\[
\hat{L}^2 |l, m\rangle = \hbar^2 l(l+1) |l, m\rangle, \quad \hat{L}_z |l, m\rangle = \hbar m |l, m\rangle,
\]

gdzie \(l = 0, 1, 2, \ldots\), a \(m = -l, -l+1, \ldots, l\).

Uogólniony operator momentu pędu \(\hat{\mathbf{J}} = \hat{\mathbf{L}} + \hat{\mathbf{S}}\) łączy moment pędu orbitalny \(\hat{\mathbf{L}}\) oraz spin \(\hat{\mathbf{S}}\).

Operator spinu \(\hat{\mathbf{S}}\) opisuje wewnętrzny moment pędu cząstek, niezwiązany z ruchem orbitalnym. Składowe spinu również spełniają relacje komutacyjne analogiczne do momentu pędu:

\[
[\hat{S}_i, \hat{S}_j] = i \hbar \epsilon_{ijk} \hat{S}_k.
\]

Dla spinu \(s\) (np. \(s = \frac{1}{2}\) dla elektronu), własne wartości operatorów \(\hat{S}^2\) i \(\hat{S}_z\) są:

\[
\hat{S}^2 |s, m_s\rangle = \hbar^2 s(s+1) |s, m_s\rangle, \quad \hat{S}_z |s, m_s\rangle = \hbar m_s |s, m_s\rangle,
\]

gdzie \(m_s = -s, -s+1, \ldots, s\).

Własne funkcje momentu pędu i spinu tworzą bazę przestrzeni stanów kwantowych, na której można opisywać stan cząstki z uwzględnieniem zarówno ruchu orbitalnego, jak i spinu.

\subsection{Atom wodoru: stany własne oraz energie własne}


Atom wodoru opisuje równanie Schrödingera z potencjałem Coulomba:

\[
\hat{H} = -\frac{\hbar^2}{2m} \nabla^2 - \frac{e^2}{4 \pi \varepsilon_0 r},
\]

gdzie \(m\) to masa elektronu, \(e\) ładunek elementarny, a \(r\) odległość elektronu od jądra.

Stany własne \(|n, l, m\rangle\) są jednocześnie własnymi funkcjami operatorów:

\[
\hat{H}|n, l, m\rangle = E_n |n, l, m\rangle,
\]

\[
\hat{L}^2 |n, l, m\rangle = \hbar^2 l(l+1) |n, l, m\rangle,
\]

\[
\hat{L}_z |n, l, m\rangle = \hbar m |n, l, m\rangle,
\]

gdzie liczby kwantowe przyjmują wartości

\[
n = 1, 2, 3, \ldots, \quad l = 0, 1, \ldots, n-1, \quad m = -l, -l+1, \ldots, l.
\]

Energia własna jest określona wzorem:

\[
E_n = - \frac{m e^4}{2 (4 \pi \varepsilon_0)^2 \hbar^2} \frac{1}{n^2} = - \frac{13.6\,\mathrm{eV}}{n^2}.
\]

Energia zależy wyłącznie od głównej liczby kwantowej \(n\), co prowadzi do degeneracji poziomów energetycznych względem liczb \(l\) i \(m\).

\subsection{Własności funkcji falowych bozonów oraz fermionów}


W mechanice kwantowej funkcje falowe bozonów i fermionów różnią się symetrią względem zamiany dwóch identycznych cząstek:

\begin{itemize}
    \item \textbf{Bozony} mają symetryczne funkcje falowe, tzn. przy zamianie cząstek
    \[
    \Psi(\ldots, \mathbf{r}_i, \ldots, \mathbf{r}_j, \ldots) = + \Psi(\ldots, \mathbf{r}_j, \ldots, \mathbf{r}_i, \ldots).
    \]
    
    \item \textbf{Fermiony} mają antysymetryczne funkcje falowe, tzn.
    \[
    \Psi(\ldots, \mathbf{r}_i, \ldots, \mathbf{r}_j, \ldots) = - \Psi(\ldots, \mathbf{r}_j, \ldots, \mathbf{r}_i, \ldots).
    \]
\end{itemize}

Antysymetria funkcji falowej fermionów prowadzi do zasady Pauliego wykluczania, która zabrania zajmowania tego samego stanu kwantowego przez dwie identyczne fermiony.

Symetria funkcji falowej bozonów pozwala na zajmowanie tego samego stanu kwantowego przez wiele cząstek, co jest podstawą efektów takich jak kondensacja Bosego-Einsteina.
